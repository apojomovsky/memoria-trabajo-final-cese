\chapter{Introducción Específica}

\label{Capitulo2}

En este capítulo se desglosan las diferentes herramientas tanto de hardware como software, elegidas para el desarrollo del robot propuesto.

\section{Robot Operating System}

Típicamente denominado ROS, es un framework de robótica de código abierto, el cual fue diseñado originalmente para robots de uso académico. Sin embargo, al día de hoy su uso se ha extendido tanto a la industria como al público aficionado.\newline
ROS ofrece un variado set de herramientas que facilitan las tareas del roboticista en tareas tales como paso de mensajes, computación distribuida e implementación de algoritmos para aplicaciones robóticas.

\subsection{Organización}

Sería adecuado considerar a ROS como algo más que un framework de desarrollo y referirnos a el como un meta-sistema-operativo, ya que ofrece no solo herramientas y librerías sino también funciones similares a las de un sistema operativo tales como abstracción de hardware, manejo de paquetes y un completo \textit{toolchain} de compilación. Tal como un sistema operativo "real", los archivos de ROS estan organizados de una manera particular, como se muestra en la figura \ref{fig:rosSistemaDeArchivos}.

\begin{figure}[ht]
    \centering
    \def\svgwidth{300pt}
    \input{./Figures/estructura_archivos_ros.pdf_tex}
    \caption{Organización de archivos en ROS}
    \label{fig:rosSistemaDeArchivos}
\end{figure}

% Force image to be printed
% \clearpage

\subsection{Herramienta RViz}
\subsection{Formato universal de descripción de robots URDF}
\subsection{Librería rosserial}
\subsection{Paquete de navegación ros\_navigation\_stack}
\section{iRobot Roomba 500}
\subsection{Consideraciones}
\section{Placa de desarrollo STM32-NUCLEO}
\subsection{Descripción}
\subsection{Consideraciones}
\section{Sensor Kinect 360}
\subsection{Descripción}
\subsection{Comparación con otras cámaras de profundidad}
\section{Unidad de medición inercial MPU6050}
\subsection{Descripción}
\subsection{Comparación con otras IMU}
