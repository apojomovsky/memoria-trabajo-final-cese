\chapter{Introducción General}

\label{Capítulo1}
\label{IntroducciónGeneral}


\newcommand{\keyword}[1]{\textbf{#1}}
\newcommand{\tabhead}[1]{\textbf{#1}}
\newcommand{\code}[1]{\texttt{#1}}
\newcommand{\file}[1]{\texttt{\bfseries#1}}
\newcommand{\option}[1]{\texttt{\itshape#1}}
\newcommand{\grados}{$^{\circ}$}

En este capítulo se introduce al campo de estudio de la robótica móvil. Se aborda una comparativa entre diferentes plataformas didácticas comerciales y se exponen el alcance y las motivaciones que llevaron al desarrollo del presente proyecto.

\section{Robótica móvil}

La robótica móvil se encarga del estudio de los robots móviles, haciendo especial hincapié en el desarrollo de capacidades permitan a lo mismos, decidir de manera autónoma cómo, cuándo y a dónde moverse.\newline
En contraste con los robots manipuladores cuya base se encuentra fija con respecto a un sistema de referencia, los robots móviles son aquellos capaces de moverse a si mismos de un lugar a otro. Esta particularidad obliga a que los robots móviles deban ser capaces de interactuar con entornos no determinísticos, es decir, propensos a situaciones impredecibles como por ejemplo, una puerta entreabierta, un objeto o persona obstaculizando el camino, etc.\newline

\subsection{Tipos de robots móviles}

Dependiendo de cómo realizan su locomoción, es posible caracterizar a los robots móviles en los siguientes tipos:
\begin{itemize}
	\item{Robots con patas}
	\item{Robots aéreos}
	\item{Robots con ruedas}
\end{itemize}

Cada uno de estos tipos plantea su propio set de ventajas y desventajas, así como dificultades para su implementación. En el presente trabajo se hará énfasis solo en el último tipo de la lista, es decir en robots con ruedas.

\section{Estado del arte}

Existe una amplia gama de robots móviles con ruedas ofrecidos específicamente para el sector académico. A continuación se ofrece un breve sumario de opciones que se encuentran actualmente en el mercado.
\subsection{TurtleBot}

TurtleBot constituye al día de hoy una familia de robots móviles para uso personal de bajo costo. Su uso se extiende tanto a la academia como a roboticistas aficionados en todo el mundo.\newline
Aunque su primera iteración vió la luz en 2010 con un set de características bastante modestas, el lanzamiento de nuevas versiones le aseguró su lugar como dispositivo de referencia para la plataforma ROS. \newline
Muchas de las características de diseño de éste robot se tomaron como referencia para el presente trabajo.

\subsection{Clearpath Jackal UGV}

El Jackal es un robot móvil 4x4 apto para uso en exteriores. Su robustez lo hacen la elección preferida de muchas universidades a la hora de implementar soluciones "de campo", principalmente debido a su resistencia total al polvo y al agua de lluvia. Posee una capacidad de carga de hasta 20 kg, lo que lo hace apto para cargar una importante cantidad de sensores, actuadores y manipuladores.

\subsection{Fetch Freight Base}

Fetch Robotics ofrece con el Freight Base una base robótica para uso en interiores. La misma fue diseñada específicamente para moverse en edificios adaptados a personas en sillas de ruedas que cumplen con la normativa ADA o \textit{Americans with Disabilities Act}. Gracias a esto, es posible saber de antemano si dicho robot puede utilizarse para recorrer las distintas dependencias de un edificio sin la necesidad de hacer pruebas preliminares \textit{in-situ}.

\subsection{Festo Robotino}


\subsection{Pioneer 3-DX}

\section{Motivación}


\section{Objetivos}



\section{Alcance}

