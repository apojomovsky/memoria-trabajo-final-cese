% !TEX root = ../memoria.tex

\chapter{Introducción general}

\label{Chapter1}

En este capítulo se introduce el campo de estudio de la robótica móvil y la importancia de una planta de pruebas física como motivación para la realización de este trabajo. Asimismo, se presentan los objetivos y el alcance del presente proyecto.

%----------------------------------------------------------------------------------------
%	SECTION 1
%----------------------------------------------------------------------------------------

\section{Motivación}

La robótica de manipuladores, también llamados brazos robóticos, se han ganado su puesto como ciudadanos de primera clase en la industria de la manufactura. Difícilmente podríamos al día de hoy, imaginarnos una planta de fabricación en serie que no disponga de estos dispositivos para la realización de tareas repetitivas y de alta precisión.

En la industria electrónica, por citar un ejemplo, los manipuladores son capaces de colocar componentes de montaje superfical con una precisión y velocidad por lejos sobre-humana, haciendo posible la elaboración de teléfonos celulares, computadoras portátiles, etc.
Sin embargo, y a pesar de su innegable éxito, estos robots sufren de una desventaja particular: la falta de movilidad. Un manipulador fijo posee un rango de movimiento limitado que depende del sitio en que el mismo se encuentre instalado. Por el contrario, un robot móvil sería capaz de moverse a través de la planta, permitiendo el aprovechamiento de sus facultades donde sea que estas sean precisadas.

La robótica móvil responde a la pregunta de "¿cómo moverse desde un determinado punto en el espacio a otro atravezando un entorno impredecible y sin supervisión?". Si bien la respuesta a esta pregunta podría resultar casi trivial si se tratase de un protagonista humano, la realidad es que representa un desafío mayor para un robot y por ello existe una rama completa de estudios dedicada a ella.

El estudio de la robótica en las universidades argentinas se encuentra mayormente avocada a la robótica de manipuladores. Esto tiene sentido desde un punto de vista de oferta/demanda en la industria local, sin embargo, el mercado internacional esta viviendo una fuerte demanda de profesionales capaces de entender y aplicar técnicas de robótica móvil.

Considerando que hasta hace solo unos años atrás los sensores y computadoras requeridos para estas tareas tenían costos altamente prohibitivos.
El estudio de robótica móvil es tan importante como el de la robótica de manipuladores.

Diversas empresas han adoptado este desafío y se encuentran, al día de hoy, trabajando en soluciones con robots móviles.
Los vehículos autónomos son un ejemplo de ello, 

\todo {Terminar la motivación}

%-----------------------------------
%	SUBSECTION 1
%-----------------------------------
\subsection{Subsection 1}

Nunc posuere quam at lectus tristique eu ultrices augue venenatis. Vestibulum ante ipsum primis in faucibus orci luctus et ultrices posuere cubilia Curae; Aliquam erat volutpat. Vivamus sodales tortor eget quam adipiscing in vulputate ante ullamcorper. Sed eros ante, lacinia et sollicitudin et, aliquam sit amet augue. In hac habitasse platea dictumst.

%-----------------------------------
%	SUBSECTION 2
%-----------------------------------

\subsection{Subsection 2}
Morbi rutrum odio eget arcu adipiscing sodales. Aenean et purus a est pulvinar pellentesque. Cras in elit neque, quis varius elit. Phasellus fringilla, nibh eu tempus venenatis, dolor elit posuere quam, quis adipiscing urna leo nec orci. Sed nec nulla auctor odio aliquet consequat. Ut nec nulla in ante ullamcorper aliquam at sed dolor. Phasellus fermentum magna in augue gravida cursus. Cras sed pretium lorem. Pellentesque eget ornare odio. Proin accumsan, massa viverra cursus pharetra, ipsum nisi lobortis velit, a malesuada dolor lorem eu neque.

%----------------------------------------------------------------------------------------
%	SECTION 2
%----------------------------------------------------------------------------------------

\section{Descripción de tecnologías}

Sed ullamcorper quam eu nisl interdum at interdum enim egestas. Aliquam placerat justo sed lectus lobortis ut porta nisl porttitor. Vestibulum mi dolor, lacinia molestie gravida at, tempus vitae ligula. Donec eget quam sapien, in viverra eros. Donec pellentesque justo a massa fringilla non vestibulum metus vestibulum. Vestibulum in orci quis felis tempor lacinia. Vivamus ornare ultrices facilisis. Ut hendrerit volutpat vulputate. Morbi condimentum venenatis augue, id porta ipsum vulputate in. Curabitur luctus tempus justo. Vestibulum risus lectus, adipiscing nec condimentum quis, condimentum nec nisl. Aliquam dictum sagittis velit sed iaculis. Morbi tristique augue sit amet nulla pulvinar id facilisis ligula mollis. Nam elit libero, tincidunt ut aliquam at, molestie in quam. Aenean rhoncus vehicula hendrerit.


%----------------------------------------------------------------------------------------
%	SECTION 3
%----------------------------------------------------------------------------------------

\section{Objetivos y alcance}

Sed ullamcorper quam eu nisl interdum at interdum enim egestas. Aliquam placerat justo sed lectus lobortis ut porta nisl porttitor. Vestibulum mi dolor, lacinia molestie gravida at, tempus vitae ligula. Donec eget quam sapien, in viverra eros. Donec pellentesque justo a massa fringilla non vestibulum metus vestibulum. Vestibulum in orci quis felis tempor lacinia. Vivamus ornare ultrices facilisis. Ut hendrerit volutpat vulputate. Morbi condimentum venenatis augue, id porta ipsum vulputate in. Curabitur luctus tempus justo. Vestibulum risus lectus, adipiscing nec condimentum quis, condimentum nec nisl. Aliquam dictum sagittis velit sed iaculis. Morbi tristique augue sit amet nulla pulvinar id facilisis ligula mollis. Nam elit libero, tincidunt ut aliquam at, molestie in quam. Aenean rhoncus vehicula hendrerit.
