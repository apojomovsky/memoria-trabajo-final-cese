% !TEX root = ../memoria.tex

\chapter{Conclusiones}

En este capítulo se describen los aportes generados con el trabajo realizado, detallando los logros obtenidos. Así también, se especifican las técnicas mediante las cuales se hizo posible la ejecución del proyecto. Por último, se deja constancia de las metas originales que no pudieron ser abordadas dentro del alcance final, identificando las propuestas de acción a futuro.

\label{Chapter5}

%----------------------------------------------------------------------------------------
%	SECTION 1
%----------------------------------------------------------------------------------------
\section{Conclusiones generales}

En este trabajo se completó la primer iteración en el ciclo de diseño e implementación de una planta de pruebas destinada a utilizarse en el aprendizaje de técnicas de robótica móvil a nivel personal o universitario.

Se pudo desarrollar una planta de pruebas completa y funcional en base a componentes disponibles en el mercado local, integrándola con el framework de robótica ROS. Se pudo desarrollar un firmware capaz de funcionar como una interfaz entre una base móvil comercial y ROS, así como interfacear los sensores necesarios con dicho framework de modo a posibilitar su uso inmediato en aplicaciones de navegación autónoma. Este desarrollo sienta las bases y estructura para una segunda iteración, dejando abiertas una serie de propuestas con mejoras que pueden implementarse en base a las necesidades que se le vayan presentando al usuario.

A lo largo del desarrollo de este trabajo final, se aplicaron extensivamente los conocimientos adquiridos durante la carrera. Mas allá de que el conjunto de asignaturas posibilitaron la apreciación de un panorama muy completo en lo que refiere a diseño y ejecución de un proyecto de sistemas embebidos, se destacan a continuación aquellas materias cuyo impacto en el desarrollo de este trabajo:


\begin{itemize}

    \item Protocolos de Comunicación: se aplicaron los conocimientos adquiridos en la utilización de protocolos UART e I2C. Así también, se implementó la librería rosserial, la cual permite ser utilizada de manera intercambiable tanto con comunicación UART como Ethernet, utilizadas durante la etapa inicial y final del proyecto, respectivamente.

    \item Sistemas Operativos de Tiempo Real (I y II): se utilizaron los conocimientos adquiridos aplicando FreeRTOS como sistema operativo para el sistema propuesto.

    \item Arquitectura de microprocesadores: resultó necesario tener conocimientos sobre la Arquitectura ARM Cortex M para la programación de la plataforma STM32 NUCLEO y el uso efectivo de sus periféricos.

    \item Programación de microprocesadores: se aplicaron las buenas prácticas de programación que fueron mostradas a lo largo de la materia. Se empleó un formato de código consistente a modo de obtener código más modular, legible y reutilizable.

    \item Gestión de Proyectos en Ingeniería: la elaboración de un Plan de Proyecto para organizar el Trabajo Final, el cual facilitó en gran manera la ejecución del mismo y evitó demoras innecesarias.

\end{itemize}

Por lo tanto, se concluye que los objetivos planteados al comienzo del trabajo han sido alcanzados satisfactoriamente, habiéndose cumplido con los criterios de aceptación del sistema final y obtenido conocimientos valiosos para la formación profesional del autor.

%----------------------------------------------------------------------------------------
%	SECTION 2
%----------------------------------------------------------------------------------------
\section{Trabajo a futuro}

Se plantea a continuación una lista de mejoras que podrían implementarse a fin de aumentar las capacidades del robot, las cuales se espera puedan ser abordadas por los usuarios que hayan decidido adoptar la plataforma, a quienes se exhorta a hacer aportes públicos al repositorio de modo a que todos puedan verse beneficiados con los mismos:

\begin{itemize}

    \item Agregar soporte para sensores y actuadores disponibles en el robot Roomba que aún no han sido implementados en el sistema.
    \item Agregar un magnetómetro a la IMU via SPI o bien, reemplazarla por una IMU que ofrezca 9 grados de libertad, como la MPU9250, de modo a que la estimación de la pose del robot sea mas precisa.
    \item Utilizar la salida de potencia del controlador del motor de la aspiradora para alimentar el sensor Kinect, eliminando la necesidad de una batería externa.
    \item Reemplazar el actual computador OrangePI PC con un Nvidia Jetson Nano o similar, de modo a poder procesar la información proveniente desde una mayor cantidad de sensores.
    \item Implementar un segundo sensor Kinect apuntando en dirección contraria al actual para duplicar el campo de visión.
    \item Agregar soporte para ROS 2.

\end{itemize}
